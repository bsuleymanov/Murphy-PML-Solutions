We will first consider one-dimensional “rectangles” (i.e., lines); since the dimensions are independent, we can
easily generalize to 2d. \\
For convenience, we will follow the notation of Josh Tenenbaum’s PhD thesis. In particular, let $h = \boldsymbol\theta$
be the unknown hypothesis or parameter vector, $\mathcal{H}$ be the set of possible hypotheses (rectangles), $\mathcal{H}_y$ be the
set of hypotheses consistent with observation $y$ (so the rectangles have to be big enough to capture $y$), and
$\mathcal{H}_{\mathcal{D}, y}$ be the set of hypotheses consistent with all the examples in $\mathcal{D}$ as well as with $y$. \\
The posterior predictive is given by $p(y|\mathcal{D})=\frac{p(y, \mathcal{D})}{p(\mathcal{D})}$, where 

\begin{flalign}
p(\mathcal{D}) = \int_{h \in \mathcal{H}}p(h)p(\mathcal{D}|h)dh
= \int_{h \in \mathcal{H}}p(h)/|h|^Ndh
\end{flalign} where we used the fact that $p(\mathcal{D}|h)=1/|h|^N$ if $h \in \mathcal{H}_\mathcal{D}$ and is 0 otherwise. Similarly, $p(y, \mathcal{D}) = \int_{h \in \mathcal{H}_{\mathcal{D}, y}}p(h)/|h|^Ndh$. \\
To derive the integral in Equation 1, let us assume the maximum observed value is 0 (we can pick any
maximum and recenter the data, since we assume a translation invariant prior). Then the right edge of the
rectangle must lie past the data, so $l \ge 0$. Also, if $r$ is the range spanned by the examples, then the left most
data point is at $-r$, so the left side of the rectangle must satisfy $l - s \le r$, where $s$ is size of the rectangle. \\ \\
a. Using these assumptions, show that

\begin{align*}
p(\mathcal{D}) = \frac{1}{N(N-1)r^{N-1}} 
\end{align*}
\\
Hint: use integration by parts
\begin{align*}
I = \int_{a}^{b} f(x)g'(x)dx = [f(x)g(x)]_a^b - \int_a^b f'(x)g(x)dx
\end{align*}
\\
b. To compute $p(y, \mathcal{D})$, we just need to extend the range from $r$ to $r + d$, where $d$ is the distance from $y$ to
the closest observed example. Hence show that
\begin{align*}
p(y|\mathcal{D}) = \frac{1}{(1+d/r)^{N-1}}
\end{align*}
\\ \\
\textbf{Solution} \\
a.
\begin{align*}
p(\mathcal{D}) &= \int_{h \in \mathcal{H}}p(h)/|h|^Ndh \\
&= \int_{s=r}^{\infty}\left[\int_{l=0}^{s-r}p(s)/s^ndl\right]ds \\
&= \int_{s=r}^{\infty}\frac{1}{s^{n+1}}(s-r)ds \\
&= \int_{s=r}^{\infty}\frac{s-r}{s^{n+1}}ds
\end{align*}
\\
Let's substitute: 
\begin{align*}
    f(s) &= s - r \\
    f'(s) &= 1 \\
    g'(s) &= s^{-n-1} \\
    g(s) &= \frac{s^{-n}}{-n}
\end{align*}
By integrating by parts we have
\begin{align*}
    p(\mathcal{D}) &= \left[\frac{(s-r)s^{-n}}{-n}\right]_r^\infty - \int_r^\infty \frac{s^{-n}}{-n}ds \\
    &= \left[\frac{s^{-n+1}}{-n} + \frac{rs^{-n}}{n} + \frac{1}{n}\frac{s^{-n+1}}{-n+1}\right]_r^\infty \\
    &= \frac{r^{-n+1}}{n} - \frac{rr^{-n}}{n} + \frac{r^{-n+1}}{n(n-1)} \\
    &= \frac{1}{nr^{n-1}} - \frac{r}{nr^{n-1}r} + \frac{1}{n(n-1)r^{n-1}} \\
    &= \frac{1}{n(n-1)r^{n-1}}
\end{align*} \\
b. Let $d$ be the distance from $y$ to the closest observed example if $y \notin \mathcal{D}$ and $0$ else:
\begin{align*}
p(y|\mathcal{D}) &= \frac{p(\mathcal{D}, y)}{p(\mathcal{D})} \\
&= \frac{\int_{h \in \mathcal{H}, y}p(h)/|h|^Ndh}{\int_{h \in \mathcal{H}}p(h)/|h|^Ndh} \\
&= \frac{\frac{1}{n(n-1)r^{n-1}}}{\frac{1}{n(n-1)(r+d)^{n-1}}} \\
&= \frac{r^{n-1}}{(r+d)^{n-1}} \\
&= \frac{1}{(1 + d/r)^{n-1}}
\end{align*}